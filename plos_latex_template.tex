% Template for PLoS
% Version 3.5 March 2018
%
% % % % % % % % % % % % % % % % % % % % % %
%
% -- IMPORTANT NOTE
%
% This template contains comments intended 
% to minimize problems and delays during our production 
% process. Please follow the template instructions
% whenever possible.
%
% % % % % % % % % % % % % % % % % % % % % % % 
%
% Once your paper is accepted for publication, 
% PLEASE REMOVE ALL TRACKED CHANGES in this file 
% and leave only the final text of your manuscript. 
% PLOS recommends the use of latexdiff to track changes during review, as this will help to maintain a clean tex file.
% Visit https://www.ctan.org/pkg/latexdiff?lang=en for info or contact us at latex@plos.org.
%
%
% There are no restrictions on package use within the LaTeX files except that 
% no packages listed in the template may be deleted.
%
% Please do not include colors or graphics in the text.
%
% The manuscript LaTeX source should be contained within a single file (do not use \input, \externaldocument, or similar commands).
%
% % % % % % % % % % % % % % % % % % % % % % %
%
% -- FIGURES AND TABLES
%
% Please include tables/figure captions directly after the paragraph where they are first cited in the text.
%
% DO NOT INCLUDE GRAPHICS IN YOUR MANUSCRIPT
% - Figures should be uploaded separately from your manuscript file. 
% - Figures generated using LaTeX should be extracted and removed from the PDF before submission. 
% - Figures containing multiple panels/subfigures must be combined into one image file before submission.
% For figure citations, please use "Fig" instead of "Figure".
% See http://journals.plos.org/plosone/s/figures for PLOS figure guidelines.
%
% Tables should be cell-based and may not contain:
% - spacing/line breaks within cells to alter layout or alignment
% - do not nest tabular environments (no tabular environments within tabular environments)
% - no graphics or colored text (cell background color/shading OK)
% See http://journals.plos.org/plosone/s/tables for table guidelines.
%
% For tables that exceed the width of the text column, use the adjustwidth environment as illustrated in the example table in text below.
%
% % % % % % % % % % % % % % % % % % % % % % % %
%
% -- EQUATIONS, MATH SYMBOLS, SUBSCRIPTS, AND SUPERSCRIPTS
%
% IMPORTANT
% Below are a few tips to help format your equations and other special characters according to our specifications. For more tips to help reduce the possibility of formatting errors during conversion, please see our LaTeX guidelines at http://journals.plos.org/plosone/s/latex
%
% For inline equations, please be sure to include all portions of an equation in the math environment.  For example, x$^2$ is incorrect; this should be formatted as $x^2$ (or $\mathrm{x}^2$ if the romanized font is desired).
%
% Do not include text that is not math in the math environment. For example, CO2 should be written as CO\textsubscript{2} instead of CO$_2$.
%
% Please add line breaks to long display equations when possible in order to fit size of the column. 
%
% For inline equations, please do not include punctuation (commas, etc) within the math environment unless this is part of the equation.
%
% When adding superscript or subscripts outside of brackets/braces, please group using {}.  For example, change "[U(D,E,\gamma)]^2" to "{[U(D,E,\gamma)]}^2". 
%
% Do not use \cal for caligraphic font.  Instead, use \mathcal{}
%
% % % % % % % % % % % % % % % % % % % % % % % % 
%
% Please contact latex@plos.org with any questions.
%
% % % % % % % % % % % % % % % % % % % % % % % %

\documentclass[10pt,letterpaper]{article}
\usepackage[top=0.85in,left=2.75in,footskip=0.75in]{geometry}

% amsmath and amssymb packages, useful for mathematical formulas and symbols
\usepackage{amsmath,amssymb}

% Use adjustwidth environment to exceed column width (see example table in text)
\usepackage{changepage}

% Use Unicode characters when possible
\usepackage[utf8x]{inputenc}

% textcomp package and marvosym package for additional characters
\usepackage{textcomp,marvosym}

% cite package, to clean up citations in the main text. Do not remove.
\usepackage{cite}

% Use nameref to cite supporting information files (see Supporting Information section for more info)
\usepackage{nameref,hyperref}

% line numbers
\usepackage[right]{lineno}

% ligatures disabled
\usepackage{microtype}
\DisableLigatures[f]{encoding = *, family = * }

% color can be used to apply background shading to table cells only
\usepackage[table]{xcolor}

% array package and thick rules for tables
\usepackage{array}

% create "+" rule type for thick vertical lines
\newcolumntype{+}{!{\vrule width 2pt}}

% create \thickcline for thick horizontal lines of variable length
\newlength\savedwidth
\newcommand\thickcline[1]{%
  \noalign{\global\savedwidth\arrayrulewidth\global\arrayrulewidth 2pt}%
  \cline{#1}%
  \noalign{\vskip\arrayrulewidth}%
  \noalign{\global\arrayrulewidth\savedwidth}%
}

% \thickhline command for thick horizontal lines that span the table
\newcommand\thickhline{\noalign{\global\savedwidth\arrayrulewidth\global\arrayrulewidth 2pt}%
\hline
\noalign{\global\arrayrulewidth\savedwidth}}


% Remove comment for double spacing
%\usepackage{setspace} 
%\doublespacing

% Text layout
\raggedright
\setlength{\parindent}{0.5cm}
\textwidth 5.25in 
\textheight 8.75in

% Bold the 'Figure #' in the caption and separate it from the title/caption with a period
% Captions will be left justified
\usepackage[aboveskip=1pt,labelfont=bf,labelsep=period,justification=raggedright,singlelinecheck=off]{caption}
\renewcommand{\figurename}{Fig}

% Use the PLoS provided BiBTeX style
\bibliographystyle{plos2015}
% Remove brackets from numbering in List of References
\makeatletter
\renewcommand{\@biblabel}[1]{\quad#1.}
\makeatother



% Header and Footer with logo
\usepackage{lastpage,fancyhdr,graphicx}
\usepackage{epstopdf}
%\pagestyle{myheadings}
\pagestyle{fancy}
\fancyhf{}
%\setlength{\headheight}{27.023pt}
%\lhead{\includegraphics[width=2.0in]{PLOS-submission.eps}}
\rfoot{\thepage/\pageref{LastPage}}
\renewcommand{\headrulewidth}{0pt}
\renewcommand{\footrule}{\hrule height 2pt \vspace{2mm}}
\fancyheadoffset[L]{2.25in}
\fancyfootoffset[L]{2.25in}
\lfoot{\today}

%% TEMPORARY FIGURES (REMOVE BEFORE SUBMISSION)
\usepackage[svgpath=./figs/]{svg}
\usepackage{graphicx}
\graphicspath{ {./figs/} }
\usepackage{float}
%% Include all macros below

\newcommand{\lorem}{{\bf LOREM}}
\newcommand{\ipsum}{{\bf IPSUM}}

%% END MACROS SECTION


\begin{document}
\vspace*{0.2in}

% Title must be 250 characters or less.
\begin{flushleft}
{\Large
\textbf\newline{Beta series correlations} % Please use "sentence case" for title and headings (capitalize only the first word in a title (or heading), the first word in a subtitle (or subheading), and any proper nouns).
}
\newline
% Insert author names, affiliations and corresponding author email (do not include titles, positions, or degrees).
\\
James Kent\textsuperscript{1*},
Michelle Voss\textsuperscript{1},
%Name3 Surname\textsuperscript{2,3\textcurrency},
%Name4 Surname\textsuperscript{2},
%Name5 Surname\textsuperscript{2\ddag},
%Name6 Surname\textsuperscript{2\ddag},
%Name7 Surname\textsuperscript{1,2,3*},
%with the Lorem Ipsum Consortium\textsuperscript{\textpilcrow}
\\
\bigskip
\textbf{1} Psychology Department, University of Iowa, Iowa City, Iowa, United States
%\\
%\textbf{2} Affiliation Dept/Program/Center, Institution Name, City, State, Country
%\\
%\textbf{3} Affiliation Dept/Program/Center, Institution Name, City, State, Country
%\\
\bigskip

% Insert additional author notes using the symbols described below. Insert symbol callouts after author names as necessary.
% 
% Remove or comment out the author notes below if they aren't used.
%
% Primary Equal Contribution Note
%\Yinyang These authors contributed equally to this work.

% Additional Equal Contribution Note
% Also use this double-dagger symbol for special authorship notes, such as senior authorship.
%\ddag These authors also contributed equally to this work.

% Current address notes
% \textcurrency Current Address: Dept/Program/Center, Institution Name, City, State, Country % change symbol to "\textcurrency a" if more than one current address note
% \textcurrency b Insert second current address 
% \textcurrency c Insert third current address

% Deceased author note
% \dag Deceased

% Group/Consortium Author Note
% \textpilcrow Membership list can be found in the Acknowledgments section.

% Use the asterisk to denote corresponding authorship and provide email address in note below.
* james-kent@uiowa.edu

\end{flushleft}
% Please keep the abstract below 300 words
\section*{Abstract}
Simulations, task, and combination

% Please keep the Author Summary between 150 and 200 words
% Use first person. PLOS ONE authors please skip this step. 
% Author Summary not valid for PLOS ONE submissions.   
% \section*{Author summary}
% Lorem ipsum dolor sit amet, consectetur adipiscing elit. Curabitur eget porta erat. Morbi consectetur est vel gravida pretium. Suspendisse ut dui eu ante cursus gravida non sed sem. Nullam sapien tellus, commodo id velit id, eleifend volutpat quam. Phasellus mauris velit, dapibus finibus elementum vel, pulvinar non tellus. Nunc pellentesque pretium diam, quis maximus dolor faucibus id. Nunc convallis sodales ante, ut ullamcorper est egestas vitae. Nam sit amet enim ultrices, ultrices elit pulvinar, volutpat risus.

\linenumbers

% Use "Eq" instead of "Equation" for equation citations.
\section*{Introduction}

The field of human imaging neuroscience has moved towards interrogating networks of brain regions
over individual brain regions.
Broadly defined, a brain network is a collection of widespread regions who share information.
Brain networks have primarily been investigated while participants are at rest or while they are performing
a single task for a block of time, such as tapping a finger for a 30 second period.
However, networks have resisted measurement while participants perform interleaved tasks or when
events of interest occur close together in a task, such as a task switching task where participants
must either classify a number as high/low or as odd/even depending on the cue.
In other words, brain networks have not been investigated within fast event related designs.
We seek to validate and compare two methods that can measure brain responses to events that occur
close in time.

The overarching method being investigated is beta series correlations (BSC) ~\cite{Rissman2004,Mumford2012,Turner2012a,Abdulrahman2016}
There are two main approaches to estimate the Beta Series for BSC: Least Squares All (LSA) and Least Squares Separate (LSS).
Both approaches of beta series estimation seek to derive single event estimates that represent a brain region's activity
at each individual event.
An event is defined as "a stimulus or participant response recorded during a task." ~\cite{Gorgolewski2016}
Where task is defined as "a set of structured activities performed by the participant." ~\cite{Gorgolewski2016}
The single event estimates become difficult to estimate when trials are close together,
in other words, less than 4 seconds.
Such designs are called fast event related designs.
The reason these single events are difficult to estimate in fast event related designs comes from
the sluggish Blood Oxygen Level Dependent (BOLD) response.
The BOLD response is an indirect measure of neuronal activity that take approximately 6 seconds to
peak and around 32 seconds to fully resolve.
If events are occurring (on average) 4 seconds apart, the difficulty of single event estimation
becomes apparent.
One does not know whether BOLD activity should be attributed to a target event or an
adjacent event.
LSA and LSS approach this problem differently.
LSA, namely, ignores this problem, while LSS attempts to provide a more reliable measure.

To understand how LSA and LSS tackle single event estimation, it is necessary to introduce
the General Linear Model (GLM).
GLMs are a mainstay in the neuroimaging literature, allowing researchers to model
a response that approximates the BOLD response shape and linearly scale the response
to best match the data.
The multiplicative that linearly scales the model BOLD response is known as a beta.
The beta in the GLM is where the beta in beta series comes from.
This beta is often interpreted as the amplitude of the response.
In a traditional GLM, events of the same type will be grouped together
to provide a robust estimate of whether a particular region or set of regions are
active relative to some other baseline.
That approach, however, does not tell you which regions are acting in concert
in response to the event type.
It could be the case that for half of the events of the same type two regions are active,
and for the other half, another two regions are active.
The traditional GLM will not be able differentiate the two pairs of regions.
LSS and LSA, on the other hand, propose to be able to separate the two pairs of regions.
LSA takes a variant of the traditional GLM whereby instead of providing a beta
estimate for a group of events, each event gets it's own beta estimate in a single GLM.
LSS differs from LSA by fitting a separate GLM for each event, where in each model a target
event is fit and the rest of the events are grouped together and given separate beta estimates.
the target trial estimate for each model is taken to form a beta series.
Harkening back to the sluggish BOLD response, LSA suffers when the events are close together,
since the GLM cannot reliably attribute which BOLD response should correspond to which event.
LSS attempts to reduce the model confusion by only having one single event estimate per model,
so the non-target events have their beta influenced by multiple observations, ostensibly
providing a more reliable single event beta estimate.

While previous work has simulated beta series and evaluated beta series on real data,
several key gaps remain.
One, simulations have not taken into account realistic noise structure of fMRI data.
Two, simulations have not used optimal fast event related designs.
Three, realistic expectations of signal to noise ratios have not been considered.
Four, a quantitative comparison/validation of LSA and LSS estimation for BSC has not been done.
The present work fills these gaps and provides recommendations for further research
using these methods.

\section*{Materials and methods}
\subsection*{Beta Series Correlation Simulations}
To assess the validity and power of beta series correlations,
we simulated timeseries data and convolved short (0.2 second) task onsets with double gamma functions
and added the responses to the timeseries~\cite{Glover1999,Welvaert2011}.
We used fmrisim from the brainiak toolbox to generate a two voxel timeseries
containing drift, autocorrelation, phsysiological noise,
task related movement, and scanner noise~\cite{Ellis2020}.
For all simulations the time of repetition was set at 2 seconds.
We varied the contrast-to-noise ratio (CNR) using the amplitude of the activation
divided by the standard deviation of the noise~\cite{Welvaert2013a}.
Another parameter deemed critical by previous work is the standard deviation
of the betas relative to the standard deviation of the noise~\cite{Abdulrahman2016}.

To generate realistic numbers for simulating timeseries at varying CNRs,
we used the task switch dataset.
We ran LSS/LSA on all participants to get both trial estimates of activation (i.e., betas)
as well as residuals from the model.
To calculate CNR, several steps were taken.
First, we masked the betaseries using the Schaefer and Activation atlases (See Atlas Correlation Analysis).
Second, we took the absolute value of all masked beta estimates for a participant.
Third, we took the median beta estimates over all trial volumes resulting
in a median beta estimate amplitude for all ROIs.
Fourth, we took the standard deviation of the residuals for each ROI.
Fifth, we divided the median amplitude beta estimates by the standard deviation of the residuals
for each ROI.
Sixth, we took either the mean or max CNR across all ROIs to provide a reasonable estimate
and upper bound of CNR.
Calculating AVNR followed the same procedure as CNR with the exception of the first two steps.
First, we took the standard deviation of the beta estimates, then we followed steps three through six above.

We derived CNRs of 1 and 10 as reasonable values from the dataset.
Again, using our local dataset, we measured the standard deviation of the betas
(i.e., Activation Variation) with
the standard deviation of the residuals (i.e., Noise),
and arrived at Activation Variation Noise ratios (AVNR) of 1 and 10.
We can manipulate CNR and AVNR independently to determine whether variation of the betas
impact beta series correlation's ability to recapture the true correlation.
% display example of design
% display example of simulated timeseries
The choice of onset times was varied based on average inter-event-interval (IEI's).
The IEI is the time from the previous event onset to the next event onset.
We chose average IEIs at 2, 4, 6, and 8 seconds to reflect a common range of IEIs
for fast event related design experiments
We also varied number of events, choosing 15, 30, 45, and 60 events per condition,
which also appear to be common selections for fast event related design experiments.
The optimization of selecting event onsets was done with neurodesign~\cite{Durnez2018}.
We chose A-Optimality to optimize onsets using the genetic algorithm implemented
in neurodesign, selecting onsets with an exponential distribution.
The top 20 designs were chosen from neurodesign to reduce the likelihood
that the simulations results are an artifact of idiosyncracies of the experimental design.

The beta weights for each voxel were chosen from a multivariate normal distribution
with a mean of one and with a fixed ground truth correlation between the two voxels 
(varying between 0.0-0.9) and predefined variance of either 1 or 10
for the betas within each voxel.
The beta weights were convolved with a hemodynamic response function and scaled
relative to the noise standard deviation to CNRs of 1 or 10.

100 simulations were run for each event number (15, 30, 45, 60) IEI (2, 4, 6, 8), 
CNR (1, 10), AVNR (1, 10), condition (c1, c2), and ground truth correlation
(0.0, 0.1, 0.2, 0.3, 0.4, 0.5, 0.6, 0.7, 0.8, 0.9)
resulting in 128,000 total simulations.

To measure validity for power analyses, we first ensured that we have
the expected false positive rate when there is no difference between samples.
we took two samples (n=50) with the same ground truth correlation (e.g., 0.1)
matching on every other parameter, and ran a ttest to measure if the samples
were statistically significantly different.
Since the ground truth correlation is the same between samples,
we expect a false positive rate of \%5 at an alpha of 0.05.
We repeated this process 10,000 times to measure the false positive rate.

Once the false positive rate is established, we will establish power with the same method as above,
but with samples containing different ground truth correlations.
We select two samples where one sample had a ground truth
correlation that was either 0.1 or 0.3 less than than the other sample.
Thus we detected how well correlation differences could be detected
with different task design and noise parameters.

%- Time Repetition = 2
%- duration of experiment = variable
%- intertrial interval = 2 - 8 seconds (exponential sampling)
%- response scaled peak response is 1
%- glover HRF
% https://github.com/brainiak/brainiak/blob/dc12929a99529a69da3c3242e9c9fd31c308fbb0/brainiak/utils/fmrisim.py#L722
%- the height of the voxel HRF responses was drawn from a multivariate normal distribution with mean 1 and standard deviation of 1.
% https://github.com/jdkent/betaSeriesSimulations/blob/2fc45bcbc6873d395432d01359ba492b6f3b1458/beta_sim/interfaces/fmrisim.py#L204
% https://github.com/jdkent/betaSeriesSimulations/blob/2fc45bcbc6873d395432d01359ba492b6f3b1458/beta_sim/interfaces/fmrisim.py#L170
% - noise was added through:
%     - snr: 40
%     - sfnr: 60
%     - task_sigma $1: Size of the variance of task specific noise
%     - drift_sigma 1: Size of the variance of drift noise
%     - auto_reg_sigma 0.5: Size of the variance of autoregressive
%         noise. This is an ARMA process where the AR and MA components can be
%         separately specified.
%     - physiological_sigma 1: Size of the variance of physiological
%         noise
%     - auto_reg_rho [0.5]: The coefficients of the autoregressive
%         components you are modeling
%     - ma_rho [0.0]:The coefficients of the moving average components you
%         are modeling
%- voxels correlated at set correlation values (0.1, 0.2, 0.3)
%- betas fit by least squares
%- N=100 (for a given parameter)
%- start and end of sessions were not ignored (optimal experimental design)
%- highpass filter: 1/128 hz
\subsection*{Task Switching Validation}
To validate the betaseries simulations we used an unpublished dataset
of older adults (N=61, 31 female, age=71.75(4.77), education=17.07(2.66)) performing a mixed task switching task.
21 participants were excluded for having a framewise displacement of over 0.5mm for
over a 100 volumes, resulting in a final N of 40.
The task switch task was a mixed block/event related design containing
5 blocks (2 single task blocks and 3 mixed task blocks).
There was a 30 second rest between each block.
There were 30 trials during each single trial block,
and for the 3 mixed blocks there were 48 repeat trials and 40 switch trials total.
The single tasks consisted of identifying a number between
1 and 10 excluding 5 as high/low or odd/even.
Participants were cued to which task they were performing by the color of the square
the number was presented on.
Each stimulus was presented for 1.5 seconds, and participants were allowed
to respond within 2.0 seconds of stimulus onset.
The average IEI was 3.5 seconds following an exponential distribution.


\subsection*{Null Task Data}
In addition to the task switching task, participants also completed
two 8 minute resting state runs.
We used the resting state runs as a null model for task switching.
While the task switch data had 471 volumes, each resting state run only had
240 volumes.
In order to match the length of the resting state data with the task data, we concatenated
the two resting state runs while cutting off the first 10 volumes of the second run
and interpolating 1 volume between the two runs, resulting in 471 volumes.
The interpolation helps transition the bold series from one run to the next,
analogous to interpolation performed when scrubbing high motion volumes. 
This null task data was treated the same as the task switching data for the
beta series correlation analysis.

\subsection{Scanner Parameters}
Data were collected on a 3T GE Discovery 750.
(scan parameters for anatomical images)
(scan parameters for functional images)

% the below are not the correct details, but I should those topics to make sure
% I'm including the relevant information
% gradient echo echo-planar sequence acquired with interleaved axial 4mm slices from the bottom up. The first four volumes were discarded to allow the signal to reach a steady state.
% "RepetitionTime": 2 (aka TR in seconds)
% "FlipAngle": 76 (degrees)
% "EchoTime": 0.0279 (aka TE in seconds)
% "SliceThickness": 4 (mm)
% receiver bandwidth = +- 25.0 kHz
% FOV = 22 cm × 22 cm, matrix size = 64 × 64,
% - 61 participants (31 females)
% - 65+ age: 71.754098 mean, 4.766745 std
% - education: 17.065574 mean, 2.657498 std
% - task: task switching, 5 blocks
% - single task (odd/even) (high/low) (30 trials each block)
% - 3 mixed blocks (48 repeat/ 40 switch trials)
% - 6 30 second rest blocks
% - participants had 4 practice trials immediately before beginning
% - participants practiced the task out of scanner.
% - iti average 3.5 seconds with exponential distribution
% - TR: 2 seconds, te: 30 ms

\subsection{Preparing fMRI}
Results included in this manuscript come from preprocessing performed
using \emph{fMRIPrep} 1.5.7 (\cite{fmriprep1}; \cite{fmriprep2}; RRID:SCR\_016216),
which is based on \emph{Nipype} 1.4.0
(\cite{nipype1}; \cite{nipype2}; RRID:SCR\_002502).

\begin{description}
\item[Anatomical data preprocessing]
The T1-weighted (T1w) image was corrected for intensity non-uniformity
(INU) with \texttt{N4BiasFieldCorrection} \cite{n4}, distributed with
ANTs 2.2.0 \cite[RRID:SCR\_004757]{ants}, and used as T1w-reference
throughout the workflow.
The T1w-reference was then skull-stripped with a \emph{Nipype} implementation
of the \texttt{antsBrainExtraction.sh} workflow (from ANTs), using OASIS30ANTs
as target template.
Brain tissue segmentation of cerebrospinal fluid (CSF), white-matter (WM) and
gray-matter (GM) was performed on the brain-extracted T1w using
\texttt{fast} \cite{fsl_fast} [FSL 5.0.9, RRID:SCR\_002823,][].
Brain surfaces were reconstructed using \texttt{recon-all} \cite{fs_reconall},
[FreeSurfer 6.0.1, RRID:SCR\_001847,][] and the brain mask estimated
previously was refined with a custom variation of the method to
reconcile ANTs-derived and FreeSurfer-derived segmentations of the
cortical gray-matter of Mindboggle \cite[RRID:SCR\_002438,]{mindboggle}.
Volume-based spatial normalization to one standard space (MNI152NLin2009cAsym)
was performed through nonlinear registration with \texttt{antsRegistration}
(ANTs 2.2.0), using brain-extracted versions of both T1w reference and the T1w template.
The following templates were selected for spatial normalization: \emph{ICBM 152 Nonlinear
Asymmetrical template version 2009c} {[}\cite{mni152nlin2009casym},
RRID:SCR\_008796; TemplateFlow ID: MNI152NLin2009cAsym{]}.
\item[Functional data preprocessing]
For each of the 2 BOLD runs found per subject,
the following preprocessing was performed.
First, a reference volume and its skull-stripped version were generated
using a custom methodology of \emph{fMRIPrep}.
Susceptibility distortion correction (SDC) was omitted.
The BOLD reference was then co-registered to the T1w reference using \texttt{bbregister}
(FreeSurfer) which implements boundary-based registration \cite{bbr}.
Co-registration was configured with six degrees of freedom.
Head-motion parameters with respect to the BOLD reference (transformation matrices,
and six corresponding rotation and translation parameters) are estimated before any
spatiotemporal filtering using \texttt{mcflirt} \cite[FSL 5.0.9,]{mcflirt}.
The BOLD time-series were resampled into a standard space, correspondingly
generating the following \emph{spatially-normalized, preprocessed BOLD runs}:
MNI152NLin2009cAsym.
Several confounding time-series were calculated based on the \emph{preprocessed BOLD}:
% only used a subset of the confounding variables
framewise displacement (FD) and two region-wise global signals.
FD was calculated for each functional run, using its
implementation in \emph{Nipype} \cite[following the definitions
by]{power_fd_dvars}.
The two global signals are extracted within the
cerebrospinal fluid and the white matter.
High-pass filtering the \emph{preprocessed BOLD} time-series was done using
a discrete cosine filter with 128s cut-off.
The head-motion estimates calculated in
the correction step were also placed within the corresponding confounds file. 
Frames that exceeded a threshold of 0.5 mm FD or 1.5 standardised DVARS
were annotated as motion outliers.
An additional 4 frames at the beginning of each run were also
annotated as outliers to allow the magnet to reach equilibrium.

All resamplings can be performed with \emph{a single interpolation step} by composing all the pertinent
transformations (i.e.~head-motion transform matrices, co-registrations to anatomical
and output spaces).
Gridded (volumetric) resamplings were performed using \texttt{antsApplyTransforms} (ANTs),
configured with Lanczos interpolation to minimize the smoothing effects of other kernels
\cite{lanczos}.
\end{description}

Many internal operations of \emph{fMRIPrep} use \emph{Nilearn} 0.6.1
\cite[RRID:SCR\_001362]{nilearn}, mostly within the functional
processing workflow. For more details of the pipeline, see
\href{https://fmriprep.readthedocs.io/en/latest/workflows.html}{the
section corresponding to workflows in \emph{fMRIPrep}'s documentation}.

\subsection{BetaSeries Correlations}

Results included in this manuscript come from modeling performed using
\emph{NiBetaSeries} 0.4.3 \cite{Kent2018}, which uses
\emph{Nipype} 1.4.2 \cite{Gorgolewski2011, Gorgolewski2018}.

\hypertarget{beta-series-modeling}{%
\subsubsection{Beta Series Modeling}\label{beta-series-modeling}}

The LSS models were generated for each event in
the task following the method described in \cite{Turner2012a}, using
Nistats 0.0.1b2.\\
Prior to modeling, preprocessed data were masked, and mean-scaled over
time.
Mean scaling was not applied when calculating CNR and AVNR so the
beta estimates would be in the original BOLD units.
For each event, preprocessed data were subjected to a GLM
in which the event was modeled with its own regressor, while
all other events from that condition were modeled in a second regressor,
and other conditions were modeled in their own regressors.
So if the task has the conditions switch, repeat, and single, 
a single GLM would have 4 event regressors, 1 for the target
trial, and 3 for the remaining event types.
Each event regressor was convolved with a \cite[glover hemodynamic response
function]{Glover1999}.\\
In addition to event regressors, average white matter signal, average csf signal,
cosine basis set high pass regressors, the initial four non stead state volumes, 
and motion outliers were included
in the model as calculated in the Preparing fMRI section.
AR(1) prewhitening was applied in each model to account
for temporal autocorrelation.

The LSA model was generated following the method described in
\cite[Rissman 2004]{Rissman2004}, using Nistats 0.0.1b2.

After fitting each model, the parameter estimate (i.e., beta) map
associated with the target event's regressor was retained and
concatenated into a 4D image with all other events from the same
condition, resulting in a set of N 4D images where N refers to the
number of conditions in the task. The number of volumes in each 4D image
represents the number of events in that condition.

\hypertarget{atlas-correlation-analysis}{%
\subsubsection{Atlas Correlation
Analysis}\label{atlas-correlation-analysis}}

The beta series 4D image for each condition in the task was subjected to
an region of interest to region of interest (ROI-to-ROI) correlation analysis
to produce condition-specific correlation matrices.
Two atlases were used to generate ROI-to-ROI correlation matrices.
The \cite[Schaefer Atlas (400 parcels, 17 networks)]{Schaefer2017} was
used to comprehensively cover the cortex.
An activation atlas was generated based on an F-test across conditions
to identify regions that were reliably activated for all participants.
5mm spheres were drawn around each statistical peak (20 peaks total)
to form the activation atlas.

The correlation coefficient estimator used for this
step was empirical covariance, as implemented in Nilearn 0.4.2
\cite{Abraham2014}. Correlation coefficients were converted to
normally-distributed z-values using Fisher's r-to-z conversion
\cite{Fisher1915}.

\hypertarget{software-dependencies}{%
\subsubsection{Software Dependencies}\label{software-dependencies}}

Additional libraries used in the NiBetaSeries workflow include
\emph{Pybids} 0.9.5 \cite{Yarkoni2019}, \emph{Niworkflows} 1.0.4,
\emph{Nibabel} 2.4.1, \emph{Pandas} 0.24.2 \cite{McKinney2010}, and
\emph{Numpy} 1.18.1 \cite{VanDerWalt2011, Oliphant2006}.

\subsection*{Traditional Task Switch Analysis}
The task switch bold fmriprep output in MNI152NLin2009cAsym space
was analyzed with Nistats for first and second level analyses.
we used mean white matter signal, mean cerebrospinal fluid signal,
discrete cosine basis filter (high pass filter), framewise displacement, the first four non-steady volumes, and
all identified motion outliers as regressors in the first level model for each participant
in addition to event onsets convolved with a double gamma function ~\cite{Glover1999}.
Each image was smoothed with a 6mm full-wide half-max kernel.
We derived all condition versus baseline contrasts: single, repeat, switch, as well as
additional contrasts for switch - repeat and an F-test of all task conditions.
We ignored correctness of the participant's response since this was not important to
separate the impact of condition and error processing to validate BSC.

Second level analysis took a summary of the first level results presenting which
regions were robustly activated between participants.
For each contrast, the alpha was set to 0.01 with a cluster threshold of 10 voxels using
false discovery rate error control.

\subsection*{contrast noise ratio calculation}
To generate realistic numbers for simulating timeseries at varying CNRs,
we used the task switch dataset.
We ran LSS/LSA on all participants to get both trial estimates of activation (i.e., betas)
as well as residuals from the model.
To calculate CNR, several steps were taken.
First, we masked the betaseries using the Schaefer and Activation atlases (See Atlas Correlation Analysis).
Second, we took the absolute value of all masked beta estimates for a participant.
Third, we took the median beta estimates over all trial volumes resulting
in a median beta estimate amplitude for all ROIs.
Fourth, we took the standard deviation of the residuals for each ROI.
Fifth, we divided the median amplitude beta estimates by the standard deviation of the residuals
for each ROI.
Sixth, we took either the mean or max CNR across all ROIs to provide a reasonable estimate
and upper bound of CNR.
Calculating AVNR followed the same procedure as CNR with the exception of the first two steps.
First, we took the standard deviation of the beta estimates, then we followed steps three through six above.

\subsection*{Task Switching Beta Series Correlation Analysis}

BSC generated correlation matrices for each condition (switch, repeat, single),
each method, (LSA and LSS), each data type (taskswitch and null), and each participant (N=40).
The first check performed was contrasting the taskswitch switch condition and null switch condition
to ensure BSC from a task are different than BSC from null data.
A ttest was run on each ROI-ROI pair for the Schaefer atlas, totaling 79,800 comparisons
between task and null.
The results were corrected for multiple comparisons (Benjamini/Hochberg).


We next contrasted switch - single, repeat - single, and switch - repeat, for LSS/LSA in both
task and null data.
Valid positive rates were calculated for LSA and LSS by dividing the number of positive results (p < 0.05)
in the task data with the number of positives found in the null data.


% For figure citations, please use "Fig" instead of "Figure".
% Nulla mi mi, Fig~\ref{fig1} venenatis sed ipsum varius, volutpat euismod diam. Proin rutrum vel massa non gravida. Quisque tempor sem et dignissim rutrum. Lorem ipsum dolor sit amet, consectetur adipiscing elit. Morbi at justo vitae nulla elementum commodo eu id massa. In vitae diam ac augue semper tincidunt eu ut eros. Fusce fringilla erat porttitor lectus cursus, \nameref{S1_Video} vel sagittis arcu lobortis. Aliquam in enim semper, aliquam massa id, cursus neque. Praesent faucibus semper libero.

% Place figure captions after the first paragraph in which they are cited.
% \begin{figure}[!h]
% \caption{{\bf Bold the figure title.}
% Figure caption text here, please use this space for the figure panel descriptions instead of using subfigure commands. A: Lorem ipsum dolor sit amet. B: Consectetur adipiscing elit.}
% \label{fig1}
% \end{figure}

% Results and Discussion can be combined.
\section*{Results}

\subsection*{Beta Series Correlation Simulations}
Across all ITIs, trial numbers, CNRs, and AVNRs, the false positive rate of 5\%
holds.

\begin{figure}[H]
  \centering
  \subfloat{\includesvg[width=\textwidth]{snr-1_trial_noise-0_diff-none_simplified_pwr}}

  \subfloat{\includesvg[width=\textwidth]{snr-1_trial_noise-1_diff-none_simplified_pwr}}
  \label{fig:dnone}
\end{figure}

\begin{figure}[H]
  \ContinuedFloat
  \centering
  \subfloat{\includesvg[width=\textwidth]{snr-10_trial_noise-0_diff-none_simplified_pwr}}

  \subfloat{\includesvg[width=\textwidth]{snr-10_trial_noise-1_diff-none_simplified_pwr}}
  \caption{
    LSA/LSS shows a \%5 false positive rate for all conditions.
    Each power plot represents a sample of 50 pairs of correlations (with no true difference)
    randomly pulled from a distribution of correlations 10,000 times.
  }
  \label{fig:dnone}
\end{figure}

\begin{figure}[H]
  \centering
  \subfloat{\includesvg[width=\textwidth]{snr-1_trial_noise-0_diff-small_simplified_pwr}}

  \subfloat{\includesvg[width=\textwidth]{snr-1_trial_noise-1_diff-small_simplified_pwr}}
  \label{fig:dsmall}
\end{figure}

\begin{figure}[H]
  \ContinuedFloat
  \centering
  \subfloat{\includesvg[width=\textwidth]{snr-10_trial_noise-0_diff-small_simplified_pwr}}

  \subfloat{\includesvg[width=\textwidth]{snr-10_trial_noise-1_diff-small_simplified_pwr}}
  \caption{
    Detecting a small difference (0.1 pearson's R) between conditions.
    Each power plot represents a sample of 50 pairs of correlations (with a true difference of 0.1)
    randomly pulled from a distribution of correlations 10,000 times.
    LSS and LSA perform similarly for short ITIs and long ITIs.
    The greatest difference occurs during the four second ITI,
    especially with a CNR of 10.}
  \label{fig:dsmall}
\end{figure}

\begin{figure}[H]
  \centering
  \subfloat{\includesvg[width=\textwidth]{snr-1_trial_noise-0_diff-large_simplified_pwr}}

  \subfloat{\includesvg[width=\textwidth]{snr-1_trial_noise-1_diff-large_simplified_pwr}}
  \label{fig:dlarge}
\end{figure}

\begin{figure}[H]
  \ContinuedFloat
  \centering
  \subfloat{\includesvg[width=\textwidth]{snr-10_trial_noise-0_diff-large_simplified_pwr}}

  \subfloat{\includesvg[width=\textwidth]{snr-10_trial_noise-1_diff-large_simplified_pwr}}
  \caption{
    Detecting a large difference (0.3 pearson's R) between two conditions.
    Each power plot represents a sample of 50 pairs of correlations (with a 0.3 true difference)
    randomly pulled from a distribution of correlations 10,000 times.
    LSS and LSA perform similarly for short ITIs and long ITIs when the CNR is 10.
    LSS has an advantage for fewer trials per condition when the ITI is four seconds.
    LSS also has greater power at a CNR of 1 when the ITI is either four or
    eight seconds.
    }
  \label{fig:dlarge}
\end{figure}
The power for detecting an effect varies by ITIs, number of trials, CNR, and AVNR.
increasing ITI leads to more power
increasing the number of trials leads to more power
The CNR drastically changes whether an effect is detectable or not, depending on estimation method.
Greater AVNR also improves power
create/insert figures for results


\subsection*{Task Switching Validation}

\subsubsection{Task Switch Univariate Results}
To provide evidence task switching looks normal under traditional analyses,
we did a standard univariate test across all voxels to identify regions
of activation.

\begin{figure}[H]
  \centering
  \subfloat{
    \includesvg[width=\textwidth]{stat-map-single}
  }
  \label{fig:statmaps}
\end{figure}

\begin{figure}[H]
  \ContinuedFloat
  \centering
  \subfloat{
    \includesvg[width=\textwidth]{stat-map-repeat}
  }
  \label{fig:statmaps}
\end{figure}

\begin{figure}[H]
  \ContinuedFloat
  \centering
  \subfloat{
    \includesvg[width=\textwidth]{stat-map-switch}
  }
  \label{fig:statmaps}
\end{figure}

\begin{figure}[H]
  \ContinuedFloat
  \centering
  \subfloat{
    \includesvg[width=\textwidth]{stat-map-switch-repeat}
  }
  \caption{Univariate statistical maps of average responses and the contrast "switch - repeat"}
  \label{fig:statmaps}
\end{figure}

\subsection*{Task Switch Beta Series Correlation Simulations}

\begin{figure}[H]
  \centering
  \subfloat{
    \includesvg[width=\textwidth]{diff-small_task-taskswitch_simplified_pwr}
  }
  \label{fig:taskpwr}
\end{figure}

\begin{figure}[H]
  \centering
  \ContinuedFloat
  \subfloat{
    \includesvg[width=\textwidth]{diff-large_task-taskswitch_simplified_pwr}
  }
  \caption{
    Simulating power to detect small and large differences using
    a real task design.
    Each power plot represents a sample of 61 pairs of correlations
    randomly pulled from a distribution of correlations 10,000 times.
    The correlation pairs either had a true difference 0.1 or 0.3.
    With a large CNR, the power for both LSS and LSA
    is maximal regardless of the true difference.
    With a small CNR, LSS has greater power to detect a difference.
    }
  \label{fig:taskpwr}
\end{figure}

\subsection*{Task Switching Beta Series Correlations}
We looked at task switching data from multiple perspectives to observe the performance of LSA and LSS,
as well as beta series correlations to detect an effect at all.

\subsubsection*{Task Swiching versus Rest}
\begin{figure}[H]
  \centering
  \includegraphics[width=\textwidth]{cond1-taskswitch_cond2-rest_statistical_overlap}
  \caption{
    Task versus Rest. Correlation differences with a p-value less than 0.05.
    LSS (green), LSA (blue), and their overlap (yellow) are shown.
  }
  \label{fig:taskvrest}
\end{figure}

The most lenient comparison is to compare the switch condition from task switching to resting state.
LSA has a total of 8,202 significant results, LSS has a total of 15,292,
and their overlap is 3,677.
A number of the significant correlations are undoubtably false positives, but
since we are interested in comparing LSS and LSA and not making strong theoretical claims,
we can say this provides evidence LSS is a more sensitive measure.

\subsubsection*{Switch versus Single}
Since we are comparing two conditions with task switching (instead of comparing task with rest),
we can use the rest data as a built in null to test if false positives are greater for LSA or LSS.
First we will show the switch minus single contrast for task switching.
\begin{figure}[H]
  \centering
  \includegraphics[width=\textwidth]{cond1-switch_cond2-single_statistical_overlap}
  \caption{
    Switch versus single. Correlation differences with a p-value less than 0.05.
    LSS (green), LSA (blue), and their overlap (yellow) are shown.
  }
  \label{fig:switchvsingle}
\end{figure}
LSA has a total of 2,474 significant results, LSS has a total of 3,108,
and their overlap is 151 results.

\begin{figure}[H]
  \centering
  \includegraphics[width=\textwidth]{cond1-restswitch_cond2-restsingle_statistical_overlap}
  \caption{
    Switch versus single (in the rest data). Correlation differences with a p-value less than 0.05.
    LSS (green), LSA (blue), and their overlap (yellow) are shown.
  }
  \label{fig:restswitchvsingle}
\end{figure}
LSA has a total of 2,129 significant results and LSS has a total of 1,693,
and their overlap is 54.

\subsubsection*{Repeat versus Single}
\begin{figure}[H]
  \centering
  \includegraphics[width=\textwidth]{cond1-repeat_cond2-single_statistical_overlap}
  \caption{
    Repeat versus single. Correlation differences with a p-value less than 0.05.
    LSS (green), LSA (blue), and their overlap (yellow) are shown.
  }
  \label{fig:repeatvsingle}
\end{figure}
LSA has a total of 2,851 significant results, LSS has a total of 3,838,
and their overlap is 282 results.

\begin{figure}[H]
  \centering
  \includegraphics[width=\textwidth]{cond1-restrepeat_cond2-restsingle_statistical_overlap}
  \caption{
    Repeat versus single (in the rest data). Correlation differences with a p-value less than 0.05.
    LSS (green), LSA (blue), and their overlap (yellow) are shown.
  }
  \label{fig:restrepeatvsingle}
\end{figure}
LSA has a total of 2,048 significant results and LSS has a total of 2,140,
and their overlap is 67.

\subsubsection*{Switch versus Repeat}
\begin{figure}[H]
  \centering
  \includegraphics[width=\textwidth]{cond1-switch_cond2-repeat_statistical_overlap}
  \caption{
    Switch versus repeat. Correlation differences with a p-value less than 0.05.
    LSS (green), LSA (blue), and their overlap (yellow) are shown.
  }
  \label{fig:switchvrepeat}
\end{figure}
LSA has a total of 101 significant results, LSS has a total of 1,322,
and their overlap is 3 results.

\begin{figure}[H]
  \centering
  \includegraphics[width=\textwidth]{cond1-restswitch_cond2-restrepeat_statistical_overlap}
  \caption{
    Switch versus repeat (in the rest data). Correlation differences with a p-value less than 0.05.
    LSS (green), LSA (blue), and their overlap (yellow) are shown.
  }
  \label{fig:restswitchvrepeat}
\end{figure}
LSA has a total of 339 significant results and LSS has a total of 1,205,
and their overlap is 5 results.
% Comparing The task versus the resting state shows mixed results.
% Question: DOES LSS or LSA show more differences between task and rest for both
% the atlas and motivated ROIs?
% Question: DOES LSS or LSA show more differences between task conditions (repeat - single)?
% (either using atlas definition or traditional roi derived analysis)
% Place tables after the first paragraph in which they are cited.
% \begin{table}[!ht]
% \begin{adjustwidth}{-2.25in}{0in} % Comment out/remove adjustwidth environment if table fits in text column.
% \centering
% \caption{
% {\bf Table caption Nulla mi mi, venenatis sed ipsum varius, volutpat euismod diam.}}
% \begin{tabular}{|l+l|l|l|l|l|l|l|}
% \hline
% \multicolumn{4}{|l|}{\bf Heading1} & \multicolumn{4}{|l|}{\bf Heading2}\\ \thickhline
% $cell1 row1$ & cell2 row 1 & cell3 row 1 & cell4 row 1 & cell5 row 1 & cell6 row 1 & cell7 row 1 & cell8 row 1\\ \hline
% $cell1 row2$ & cell2 row 2 & cell3 row 2 & cell4 row 2 & cell5 row 2 & cell6 row 2 & cell7 row 2 & cell8 row 2\\ \hline
% $cell1 row3$ & cell2 row 3 & cell3 row 3 & cell4 row 3 & cell5 row 3 & cell6 row 3 & cell7 row 3 & cell8 row 3\\ \hline
% \end{tabular}
% \begin{flushleft} Table notes Phasellus venenatis, tortor nec vestibulum mattis, massa tortor interdum felis, nec pellentesque metus tortor nec nisl. Ut ornare mauris tellus, vel dapibus arcu suscipit sed.
% \end{flushleft}
% \label{table1}
% \end{adjustwidth}
% \end{table}


%PLOS does not support heading levels beyond the 3rd (no 4th level headings).
% \subsection*{\lorem\ and \ipsum\ nunc blandit a tortor}
% \subsubsection*{3rd level heading} 
% Maecenas convallis mauris sit amet sem ultrices gravida. Etiam eget sapien nibh. Sed ac ipsum eget enim egestas ullamcorper nec euismod ligula. Curabitur fringilla pulvinar lectus consectetur pellentesque. Quisque augue sem, tincidunt sit amet feugiat eget, ullamcorper sed velit. Sed non aliquet felis. Lorem ipsum dolor sit amet, consectetur adipiscing elit. Mauris commodo justo ac dui pretium imperdiet. Sed suscipit iaculis mi at feugiat. 

% \begin{enumerate}
% 	\item{react}
% 	\item{diffuse free particles}
% 	\item{increment time by dt and go to 1}
% \end{enumerate}


% \subsection*{Sed ac quam id nisi malesuada congue}

% Nulla mi mi, venenatis sed ipsum varius, volutpat euismod diam. Proin rutrum vel massa non gravida. Quisque tempor sem et dignissim rutrum. Lorem ipsum dolor sit amet, consectetur adipiscing elit. Morbi at justo vitae nulla elementum commodo eu id massa. In vitae diam ac augue semper tincidunt eu ut eros. Fusce fringilla erat porttitor lectus cursus, vel sagittis arcu lobortis. Aliquam in enim semper, aliquam massa id, cursus neque. Praesent faucibus semper libero.

% \begin{itemize}
% 	\item First bulleted item.
% 	\item Second bulleted item.
% 	\item Third bulleted item.
% \end{itemize}

\section*{Discussion}

% Nulla mi mi, venenatis sed ipsum varius, Table~\ref{table1} volutpat euismod diam. Proin rutrum vel massa non gravida. Quisque tempor sem et dignissim rutrum. Lorem ipsum dolor sit amet, consectetur adipiscing elit. Morbi at justo vitae nulla elementum commodo eu id massa. In vitae diam ac augue semper tincidunt eu ut eros. Fusce fringilla erat porttitor lectus cursus, vel sagittis arcu lobortis. Aliquam in enim semper, aliquam massa id, cursus neque. Praesent faucibus semper libero~\cite{bib3}.
\subsection*{Simulation Conclusions}

\begin{itemize}
  \item LSS and LSA perform similarly at low and high ITIs
  \item LSS outperforms at 4 seconds ITI
  \item fast(ish) event related design, use LSS
  \item mention Abdulrahman2016 because I did not set the trial variance to below 1.
\end{itemize}

\subsection*{Task Switching Conclusions}
\begin{itemize}
  \item Simulations suggested LSS would improve sensitivity of results (while maintaining a \%5 false positive rate)
  \item (Especially if the CNR was low)
  \item The task versus rest contrast showed more results for LSS,
        suggesting even the most lenient was not large enough
        to have parity between LSA and LSS.
\end{itemize}

\subsection{Limitations}
\section*{Conclusion}

% CO\textsubscript{2} Maecenas convallis mauris sit amet sem ultrices gravida. Etiam eget sapien nibh. Sed ac ipsum eget enim egestas ullamcorper nec euismod ligula. Curabitur fringilla pulvinar lectus consectetur pellentesque. Quisque augue sem, tincidunt sit amet feugiat eget, ullamcorper sed velit. 

% Sed non aliquet felis. Lorem ipsum dolor sit amet, consectetur adipiscing elit. Mauris commodo justo ac dui pretium imperdiet. Sed suscipit iaculis mi at feugiat. Ut neque ipsum, luctus id lacus ut, laoreet scelerisque urna. Phasellus venenatis, tortor nec vestibulum mattis, massa tortor interdum felis, nec pellentesque metus tortor nec nisl. Ut ornare mauris tellus, vel dapibus arcu suscipit sed. Nam condimentum sem eget mollis euismod. Nullam dui urna, gravida venenatis dui et, tincidunt sodales ex. Nunc est dui, sodales sed mauris nec, auctor sagittis leo. Aliquam tincidunt, ex in facilisis elementum, libero lectus luctus est, non vulputate nisl augue at dolor. For more information, see \nameref{S1_Appendix}.

\section*{Supporting information}

% Include only the SI item label in the paragraph heading. Use the \nameref{label} command to cite SI items in the text.
% \paragraph*{S1 Fig.}
% \label{S1_Fig}
% {\bf Bold the title sentence.} Add descriptive text after the title of the item (optional).

% \paragraph*{S2 Fig.}
% \label{S2_Fig}
% {\bf Lorem ipsum.} Maecenas convallis mauris sit amet sem ultrices gravida. Etiam eget sapien nibh. Sed ac ipsum eget enim egestas ullamcorper nec euismod ligula. Curabitur fringilla pulvinar lectus consectetur pellentesque.

% \paragraph*{S1 File.}
% \label{S1_File}
% {\bf Lorem ipsum.}  Maecenas convallis mauris sit amet sem ultrices gravida. Etiam eget sapien nibh. Sed ac ipsum eget enim egestas ullamcorper nec euismod ligula. Curabitur fringilla pulvinar lectus consectetur pellentesque.

% \paragraph*{S1 Video.}
% \label{S1_Video}
% {\bf Lorem ipsum.}  Maecenas convallis mauris sit amet sem ultrices gravida. Etiam eget sapien nibh. Sed ac ipsum eget enim egestas ullamcorper nec euismod ligula. Curabitur fringilla pulvinar lectus consectetur pellentesque.

% \paragraph*{S1 Appendix.}
% \label{S1_Appendix}
% {\bf Lorem ipsum.} Maecenas convallis mauris sit amet sem ultrices gravida. Etiam eget sapien nibh. Sed ac ipsum eget enim egestas ullamcorper nec euismod ligula. Curabitur fringilla pulvinar lectus consectetur pellentesque.

% \paragraph*{S1 Table.}
% \label{S1_Table}
% {\bf Lorem ipsum.} Maecenas convallis mauris sit amet sem ultrices gravida. Etiam eget sapien nibh. Sed ac ipsum eget enim egestas ullamcorper nec euismod ligula. Curabitur fringilla pulvinar lectus consectetur pellentesque.

\section*{Acknowledgments}
%Cras egestas velit mauris, eu mollis turpis pellentesque sit amet. Interdum et malesuada fames ac ante ipsum primis in faucibus. Nam id pretium nisi. Sed ac quam id nisi malesuada congue. Sed interdum aliquet augue, at pellentesque quam rhoncus vitae.

\nolinenumbers

\bibliography{plos}
% Either type in your references using
% \begin{thebibliography}{}
% \bibitem{}
% Text
% \end{thebibliography}
%
% or
%
% Compile your BiBTeX database using our plos2015.bst
% style file and paste the contents of your .bbl file
% here. See http://journals.plos.org/plosone/s/latex for 
% step-by-step instructions.
% 
% \begin{thebibliography}{10}

% \bibitem{bib1}
% Conant GC, Wolfe KH.
% \newblock {{T}urning a hobby into a job: how duplicated genes find new
%   functions}.
% \newblock Nat Rev Genet. 2008 Dec;9(12):938--950.

% \bibitem{bib2}
% Ohno S.
% \newblock Evolution by gene duplication.
% \newblock London: George Alien \& Unwin Ltd. Berlin, Heidelberg and New York:
%   Springer-Verlag.; 1970.

% \bibitem{bib3}
% Magwire MM, Bayer F, Webster CL, Cao C, Jiggins FM.
% \newblock {{S}uccessive increases in the resistance of {D}rosophila to viral
%   infection through a transposon insertion followed by a {D}uplication}.
% \newblock PLoS Genet. 2011 Oct;7(10):e1002337.

% \end{thebibliography}



\end{document}
